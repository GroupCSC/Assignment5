\documentclass[12pt]{article}
\begin{document}

\section*{\centerline{AN ENVIRONMENT BLOG}}

\subsection*{1. Introduction}
\subsubsection*{1.1. Background}
Uganda mostly depends on tourism for revenue and the most of the tourism is environmental, there are also other tourist attractions like the night life, leisure centers like hotels and religious sights. Uganda has also beautiful weather and is home to mountain gorillas which are the main tourist attractions in the country
Poaching was mainly rampant in the 1970,s during amins regime where poaching was  a sport, this led to the extinction of rhinos in Uganda, they were later introduced into Uganda and are kept in special reserves 

\subsubsection*{1.2. Problem statement}
The increase in environmental degradation has led more tourist centers like national parks to suffer due to encroaching communities that are destroying the habitats and increasing the conflicts between the animals and communities, other issues like poaching have also led to reduction in numbers of animals.
Pouching has also been a continous problem especially in queen elizabeth
 where most of the poaching happens due to animals crossing to DRC
where armed groups hunt them for money to support their activities

There are also cases of the locals competing with space and some times
 encroachment by both people and animals which lead to more animals getting killed

The blog app we want to make would enable the people to raise awareness and more to the masses about the dangers facing our much-needed tourism sector

\subsection*{1.3. Objectives}
\begin{itemize}
\item To reduce environmental degradation

\item To promote the tourism sector locally

\item To raise awareness about poaching

\item To reduce the number of animals being poached reducing the chances of extinction

\item Protect the major source of revenue to the country(tourism)
\end{itemize}

\subsection*{2. Literature review}
\subsubsection*{2.1. Introduction}
There are many ways poaching in Uganda and africa can be reduced and people have suggested some awys to achieve this goal

\subsubsection*{2.2. Findings}
The push for a global ban of international and domestic markets should be seen as a policy experiment. It may work to reduce poaching which will be a tremendous outcome for Africa’s elephants. But the conservation community needs to make sure that this stronger ban is not just rhetoric. The impact of actions like the continued ban on international trade and the closure of the Chinese and other domestic ivory markets need to be monitored, and measured.[1]

From 2003 to 2010 the WWF,s work in the Virunga Forest on the borders of the Democratic Republic of Congo, Rwanda and Uganda helped increase gorillas by 26%, from 380 to 480. This happened despite the presence of violent militias in the region – a testament to the bravery and dedication of our anti-poaching patrols.[2]

\subsubsection*{2.3. Conclusion}
Overall many activities and policies have been setup to reduce the poaching in Uganda and the rest of africa with much success but they are still faced with more challenges which is the reason to work harder to achieve even better results

\subsection *{3. Research Methodology}

We used\textbf{ applied research} to identify the problem, \textbf{analytical research} to get the 
numbers of elephant deaths each year in
uganda which were roughly 3 elephants a year according 
to travelhemispheres.com and used \textbf{quantitative research} to and 
established that there are over 5000 
elephants in uganda


\subsection*{4. Sample of results}





\subsection*{5. Index}
\subsection *{5.1. Electronic Documents}
\subsection *{5.2. Article in Online Web journal}
[1] "A populist tighter ivory trade ban is not enough to save Africa’s elephants" \newline
 http://theconversation.com/a-populist-tighter-ivory-trade-ban-is-not-enough-to-save-africas-elephants-66433 \newline
[2] "Stopping poaching" \newline
 http://wwf.panda.org/?199903/Stopping-poaching
\end{document}

